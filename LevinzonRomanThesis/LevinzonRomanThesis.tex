% arara: xelatex
% arara: xelatex
% arara: xelatex


% options:
% thesis=B bachelor's thesis
% thesis=M master's thesis
% czech thesis in Czech language
% english thesis in English language
% hidelinks remove colour boxes around hyperlinks

\documentclass[thesis=B,english]{FITthesis}[2012/10/20]

% \usepackage[utf8]{inputenc} % LaTeX source encoded as UTF-8
% \usepackage[latin2]{inputenc} % LaTeX source encoded as ISO-8859-2
% \usepackage[cp1250]{inputenc} % LaTeX source encoded as Windows-1250

\usepackage{graphicx} %graphics files inclusion
% \usepackage{subfig} %subfigures
% \usepackage{amsmath} %advanced maths
% \usepackage{amssymb} %additional math symbols

\usepackage{dirtree} %directory tree visualisation

% % list of acronyms
% \usepackage[acronym,nonumberlist,toc,numberedsection=autolabel]{glossaries}
% \iflanguage{czech}{\renewcommand*{\acronymname}{Seznam pou{\v z}it{\' y}ch zkratek}}{}
% \makeglossaries

% % % % % % % % % % % % % % % % % % % % % % % % % % % % % % 
% EDIT THIS
% % % % % % % % % % % % % % % % % % % % % % % % % % % % % % 

\department{Department of \ldots (SPECIFY)}
\title{StudyPad - Android Client}
\newcommand{\appname}{StudyPad \space}
\authorGN{Roman} %author's given name/names
\authorFN{Levinzon} %author's surname
\author{Roman Levinzon} %author's name without academic degrees
\authorWithDegrees{Roman Levinzon} %author's name with academic degrees
\supervisor{Ing. Miroslav Bal{\'i}k, Ph.D}
\acknowledgements{THANKS (remove entirely in case you do not with to thank anyone)}
\abstractEN{StudyPad is a combination of note-taking service and a social network, aimed to help students to memorise different pieces of information.  The goal of this thesis is to develop an application for Android OS which will serve as client. This text acknowledges existing solutions, contains domain and requirements analysis, description and choise of application's architecture and it's implementation}


\abstractCS{StudyPad je kombinace slu{\v z}by pro pori{\v z}ovan{\' i} pozn{\'a}mek a soc{\'i}{\'a}ln{\'i} s{\'i}t{\v e} s c{\'i}lem  pomoci studentum zapamatovat si ruzn{\'e} informace. C{\'i}lem pr{\'a}ce je vyvinout aplikaci pro OS Android, kter{\'a} bude slou{\v z}it jako klient. Tento text uzn{\'a}v{\'a} st{\'a}vaj{\' i}c{\'i} {\v r}e{\v s}en{\'i}, obsahuje anal{\'y}zu dom{\'e}n a po{\v z}adavku, popis a v{\'y}b{\v e}r architektury aplikace a jej{\'i} implementace}
\placeForDeclarationOfAuthenticity{Prague}
\keywordsCS{Android, Kotlin, MVVM}
\keywordsEN{Android, Kotlin, MVVM}
\declarationOfAuthenticityOption{1} %select as appropriate, according to the desired license (integer 1-6)
% \website{http://site.example/thesis} %optional thesis URL


\begin{document}

% \newacronym{CVUT}{{\v C}VUT}{{\v C}esk{\' e} vysok{\' e} u{\v c}en{\' i} technick{\' e} v Praze}
% \newacronym{FIT}{FIT}{Fakulta informa{\v c}n{\' i}ch technologi{\' i}}

\setsecnumdepth{part}
\chapter{Introduction}



\setsecnumdepth{all}
\chapter{Analysis}
\section{System description}
\section{Existing solutions}
\section{Domain model}
\section{Android Platform}

\newpage
\section{Requirements}
It is important to establish all functional and non-functional requirements for \appname. This section contains all requirements designed before the start
of the development

\subsection{Non-functional requirements}

\begin{itemize}
  \item \textbf{N1: Native Android application}  Application will be written using native Android Sdk
  \item \textbf{N2: API 21 support} Application minimal SDK version will be 21 (a.k.a Android Lollipop)
  \item \textbf{N3: Material Design} Application user interface will follow latest Material design guidelines and best practises
  \item \textbf{N4: Scalable app architecture} Application's architecture must be scalable and easy testable
  \item \textbf{N5: Tablet \& Phone support} Application's GUI must be well suited for multiple screen sizes
  \item \textbf{N6: Multiple language support} Application will support multiple languages
\end{itemize}

\subsection{Functional requirements}
\bigskip
\textbf{User Authentication}
\begin{itemize}
	\item \textbf{F1: Registration/Login using email} Access to \appname is possible by creating an account using email address/password combination
	\item \textbf{F2: Registration/Login using Facebook} Users will be able to use their Facebook account to access \appname
	\item \textbf{F3: Registration/Login using Google} Users will be able to use their Google account to access \appname
	\item \textbf{F4: Store OAuth token} API Authentication Token will be stored in device memory
	\item \textbf{F5: Token refreshment} API Token will be refreshed when needed, so user won't have to login again
	\item \textbf{F6: University selection} As a part of user registration flow, user will be able to select his university
\end{itemize}
\bigskip
\textbf{Library Management (Notes \& Notebooks)}
\begin{itemize}
	\item \textbf{F7: Notebook creation} User will be able to create new notebooks with the name they choose
	\item \textbf{F8: Notebook deletion} User will be able to delete existing notebooks
	\item \textbf{F9: Notebook name edition} User will be able to edit notebooks names
	\item \textbf{F10: Note creation} User will be able to create a note with specific title and content.
	\item \textbf{F11: Note edition} User will be able to edit existing note, or completely delete it
	\item \textbf{F12: Show Notebooks} : User will be able to view all the notebooks he created
	\item \textbf{F13: Show Notes}: By clicking on notebook item, user will be able to view the list of notes that are assigned to this notebooks
\end{itemize}

\bigskip
\textbf{Sharing Hub}
\begin{itemize}
	\item \textbf{F14: View published notebooks} User will be able to view notebooks published by other users
	\item \textbf{F15: Search through published books} User will be able to search through the published notebooks by applying different filters (such as author, university and subject)
	\item \textbf{F16: Browse through published notebook} User will be able to see notes inside the notebook that's been published
	\item \textbf{F17: View comments} User will be able to view others users comments discussing a notebook that's been published
	\item \textbf{F18: Leave a comment} User can comment on other user published notebook
	\item \textbf{F19: Delete a comment} Application will let user to delete his/her com- ment.
	\item \textbf{F20: Save published notebook} User will be able to save published notebook to his library
	\item \textbf{F21: Publish notebook} User will be able to publish his notebook
	\item \textbf{F22: Update published notebook} Author of the published notebook will be able to update it's information
	\item \textbf{F23: Delete published notebook} Author of the published notebook will be able to delete the his notebook
	\item \textbf{F24: Share notebook} User will be able to share his notebook by generating a deeplink
 \end{itemize}

\bigskip
\textbf{Study Hub}
\begin{itemize}
	\item \textbf{F25: Start a basic self-check} User will be able to use an interactive way to look through his notes
	\item \textbf{F26: Start a written test} User will be able to participate in a written test based on one of his notebooks to test his knowledge
	\item \textbf{F27: Start a quiz} User will be able to participate in quiz challenge that will be based on one of his notebooks
	
\end{itemize}

\bigskip
\textbf{Settings}
\begin{itemize}
	\item \textbf{F28: View Profile Information} User will be able to view his profile information such as first name, last name and his university
	\item \textbf{F29: Edit Profile Information} User will be able to edit his profile information
	\item \textbf{F30: Logout} User will be able to logout from the application.
\end{itemize}



\section{Platform-Independent Model}

\chapter{Design}
\section{Wireframes}
\section{Application architecture}
\section{Platform-specific model}
\section{Main sequence diagrams}

\chapter{Implementation}
\section{Choice of technologies}
\section{Component diagram}
\section{Installation}


\chapter{Testing}


\setsecnumdepth{part}
\chapter{Conclusion}


\bibliographystyle{iso690}
\bibliography{mybibliographyfile}

\setsecnumdepth{all}
\appendix

\chapter{Acronyms}
% \printglossaries
\begin{description}
	\item[GUI] Graphical user interface
	\item[XML] Extensible markup language
\end{description}


\chapter{Contents of enclosed CD}

%change appropriately

\begin{figure}
	\dirtree{%
		.1 readme.txt\DTcomment{the file with CD contents description}.
		.1 exe\DTcomment{the directory with executables}.
		.1 src\DTcomment{the directory of source codes}.
		.2 wbdcm\DTcomment{implementation sources}.
		.2 thesis\DTcomment{the directory of \LaTeX{} source codes of the thesis}.
		.1 text\DTcomment{the thesis text directory}.
		.2 thesis.pdf\DTcomment{the thesis text in PDF format}.
		.2 thesis.ps\DTcomment{the thesis text in PS format}.
	}
\end{figure}

\end{document}
