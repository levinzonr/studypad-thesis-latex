% arara: xelatex
% arara: xelatex
% arara: xelatex


% options:
% thesis=B bachelor's thesis
% thesis=M master's thesis
% czech thesis in Czech language
% english thesis in English language
% hidelinks remove colour boxes around hyperlinks

\documentclass[thesis=B,english]{FITthesis}[2012/10/20]

% \usepackage[utf8]{inputenc} % LaTeX source encoded as UTF-8
% \usepackage[latin2]{inputenc} % LaTeX source encoded as ISO-8859-2
% \usepackage[cp1250]{inputenc} % LaTeX source encoded as Windows-1250

\usepackage{graphicx} %graphics files inclusion
% \usepackage{subfig} %subfigures
% \usepackage{amsmath} %advanced maths
% \usepackage{amssymb} %additional math symbols

\usepackage{dirtree} %directory tree visualisation

% % list of acronyms
% \usepackage[acronym,nonumberlist,toc,numberedsection=autolabel]{glossaries}
% \iflanguage{czech}{\renewcommand*{\acronymname}{Seznam pou{\v z}it{\' y}ch zkratek}}{}
% \makeglossaries

% % % % % % % % % % % % % % % % % % % % % % % % % % % % % % 
% EDIT THIS
% % % % % % % % % % % % % % % % % % % % % % % % % % % % % % 

\department{Department of \ldots (SPECIFY)}
\title{StudyPad - Android Client}
\newcommand{\appname}{StudyPad}
\authorGN{Roman} %author's given name/names
\authorFN{Levinzon} %author's surname
\author{Roman Levinzon} %author's name without academic degrees
\authorWithDegrees{Roman Levinzon} %author's name with academic degrees
\supervisor{Ing. Miroslav Bal{\'i}k, Ph.D}
\acknowledgements{THANKS (remove entirely in case you do not with to thank anyone)}
\abstractEN{StudyPad is a combination of note-taking service and a social network, aimed to help students to memorise different pieces of information.  The goal of this thesis is to develop an application for Android OS which will serve as client. This text acknowledges existing solutions, contains domain and requirements analysis, description and choise of application's architecture and it's implementation}


\abstractCS{StudyPad je kombinace slu{\v z}by pro pori{\v z}ovan{\' i} pozn{\'a}mek a soc{\'i}{\'a}ln{\'i} s{\'i}t{\v e} s c{\'i}lem  pomoci studentum zapamatovat si ruzn{\'e} informace. C{\'i}lem pr{\'a}ce je vyvinout aplikaci pro OS Android, kter{\'a} bude slou{\v z}it jako klient. Tento text uzn{\'a}v{\'a} st{\'a}vaj{\' i}c{\'i} {\v r}e{\v s}en{\'i}, obsahuje anal{\'y}zu dom{\'e}n a po{\v z}adavku, popis a v{\'y}b{\v e}r architektury aplikace a jej{\'i} implementace}
\placeForDeclarationOfAuthenticity{Prague}
\keywordsCS{Android, Kotlin, MVVM}
\keywordsEN{Android, Kotlin, MVVM}
\declarationOfAuthenticityOption{1} %select as appropriate, according to the desired license (integer 1-6)
% \website{http://site.example/thesis} %optional thesis URL


\begin{document}

% \newacronym{CVUT}{{\v C}VUT}{{\v C}esk{\' e} vysok{\' e} u{\v c}en{\' i} technick{\' e} v Praze}
% \newacronym{FIT}{FIT}{Fakulta informa{\v c}n{\' i}ch technologi{\' i}}

\setsecnumdepth{part}
\chapter{Introduction}


\appname\ is a combination of a note-taking service and a social network. Main purpose of the app is help users memorize different piecies  of information and make it easy this information easy to exchange.
To allow this, user will be able to create notes, that will act a single information piece. Notes are stored in distinct sets - Notebooks.

\setsecnumdepth{all}
\chapter{Analysis}
\section{System description}
\section{Existing solutions}
\section{Domain model}
\section{Android Platform}

\newpage
\section{Requirements}
It is important to establish all functional and non-functional requirements for \appname. Section bellow contains all requirements designed before the start
of the development

\subsection{Functional requirements}
\bigskip
\textbf{User Authentication}
\begin{itemize}
	\item \textbf{F1: Registration/Login using email} Access to \appname\ is possible by creating an account using email address/password combination.
	\item \textbf{F2: Registration/Login using Facebook} User will be able to use his/her Facebook account to access \appname.
	\item \textbf{F3: Registration/Login using Google} User will be able to use his/her Google account to access \appname.
	\item \textbf{F4: Store OAuth token} API Authentication Token will be stored in device memory.
	\item \textbf{F5: Token refreshment} API Token will be refreshed when needed, so user won't have to login again.
	\item \textbf{F6: University selection} As a part of user registration flow, user will be able to select his/her university.
\end{itemize}
\bigskip
\textbf{Library Management (Notes \& Notebooks)}
\begin{itemize}
	\item \textbf{F7: Notebook creation} User will be able to create new notebooks with the name he/she choose.
	\item \textbf{F8: Notebook deletion} User will be able to delete existing notebooks.
	\item \textbf{F9: Notebook name edition} User will be able to edit notebooks names.
	\item \textbf{F10: Note creation} User will be able to create a note with specific title and content.
	\item \textbf{F11: Note edition} User will be able to edit existing note, or completely delete it.
	\item \textbf{F12: Show Notebooks} : User will be able to view all the notebooks he/she created.
	\item \textbf{F13: Show Notes}: By clicking on notebook item, user will be able to view the list of notes that are assigned to this notebook.
\end{itemize}

\bigskip
\textbf{Sharing Hub}
\begin{itemize}
	\item \textbf{F14: View published notebooks} User will be able to view notebooks published by other users.
	\item \textbf{F15: Search through published books} User will be able to search through the published notebooks by applying different filters (such as author, university and subject/topic).
	\item \textbf{F16: Browse through published notebook} User will be able to see notes inside the notebook that's been published.
	\item \textbf{F17: View comments} User will be able to view others users comments discussing a notebook that's been published.
	\item \textbf{F18: Leave a comment} User can comment on other user published notebook.
	\item \textbf{F19: Delete a comment} Application will let user to delete his/her comment.
	\item \textbf{F20: Save published notebook} User will be able to save published notebook to his/her library.
	\item \textbf{F21: Publish notebook} User will be able to publish his notebook.
	\item \textbf{F22: Update published notebook} Author of the published notebook will be able to update it's information.
	\item \textbf{F23: Delete published notebook} Author of the published notebook will be able to delete the his/her notebook from shared space.
	\item \textbf{F24: Share notebook} User will be able to share his notebook by generating a deep-link.
 \end{itemize}

\bigskip
\textbf{Study Hub}
\begin{itemize}
	\item \textbf{F25: Start a basic self-check} User will be able to use an interactive way to look through his/her notes
	\item \textbf{F26: Start a written test} User will be able to participate in a written test based on one of notebooks to test his/her knowledge
	\item \textbf{F27: Start a quiz} User will be able to participate in quiz challenge that will be based on one of his/her notebooks
	
\end{itemize}

\bigskip
\textbf{Settings}
\begin{itemize}
	\item \textbf{F28: View Profile Information} User will be able to view his/her profile information such as first name, last name and his university.
	\item \textbf{F29: Edit Profile Information} User will be able to edit his/her profile information.
	\item \textbf{F30: Logout} User will be able to logout from the application.
\end{itemize}


\subsection{Non-functional requirements}

\begin{itemize}
  \item \textbf{N1: Native Android application}  Application will be written using native Android SDK.
  \item \textbf{N2: Android Version} Application minimal SDK version must be low enough to support as many devices as possible and high enough to use latest Android APIs considering other functional and non-functional requirements.
  \item \textbf{N3: Material Design} Application user interface will follow latest Material design guidelines and best practises.
  \item \textbf{N4: Scalable app architecture} Application's architecture must be scalable and easy testable.
  \item \textbf{N5: Tablet \& Phone support} Application GUI must be well suited for multiple screen sizes.
  \item \textbf{N6: App Localization} Application will be able to adapt to different languages based on user locale
\end{itemize}


\newpage

\section{Existing solutions}

There are several apps out there, whose goal is similar to \appname. However, most of the solutions are limited to learning languages and have limited sharing and discovering options. Table bellow shows requirementes


\bigskip
\textbf{Quizlet} is primarily used for learning languages, from where most of the limitations come from. Closest analogy to Notebook there is Study set with Terms inside. This makes it easier for tests generation, but limits user when hes trying to learn anything other than new words

\begin{itemize}
	\item \textbf{F15: View published notebooks} When searching for Notebooks, The only distinctions between Study Sets in Quizlet is its name and language. Lack of filtering options makes it hard to find something specific
	\item \textbf{F20, F21: Notebooks sharing} - All Study Sets in Quizlet are visible to everyone by default. This makes it impossible to distinct between between personal study set and a published one. Moreover, flow of saving a study set is not very straightforward
	\item \textbf{F29: } Profile Edition is fairly limited. User only allowed to change his/her username, and only once





\end{itemize}




\section{Platform-Independent Model}


\chapter{Design}
\section{Wireframes}
\section{Application architecture}
\section{Platform-specific model}
\section{Main sequence diagrams}

\chapter{Implementation}
\section{Choice of technologies}
\section{Component diagram}
\section{Installation}


\chapter{Testing}


\setsecnumdepth{part}
\chapter{Conclusion}


\bibliographystyle{iso690}
\bibliography{mybibliographyfile}

\setsecnumdepth{all}
\appendix

\chapter{Acronyms}
% \printglossaries
\begin{description}
	\item[GUI] Graphical user interface
	\item[XML] Extensible markup language
\end{description}


\chapter{Contents of enclosed CD}

%change appropriately

\begin{figure}
	\dirtree{%
		.1 readme.txt\DTcomment{the file with CD contents description}.
		.1 exe\DTcomment{the directory with executables}.
		.1 src\DTcomment{the directory of source codes}.
		.2 wbdcm\DTcomment{implementation sources}.
		.2 thesis\DTcomment{the directory of \LaTeX{} source codes of the thesis}.
		.1 text\DTcomment{the thesis text directory}.
		.2 thesis.pdf\DTcomment{the thesis text in PDF format}.
		.2 thesis.ps\DTcomment{the thesis text in PS format}.
	}
\end{figure}

\end{document}
